\documentclass{bmstu}

\usepackage{placeins}

\begin{document}
  \chapter*{Система разграничения доступа к исходному коду
  лабораторных работ}

  \section*{Глоссарий}

  \begin{itemize}
    \item[] \textbf{Валидация данных}~--- проверка данных на
      соответствие заданным условиям и ограничениям.
    \item[] \textbf{Web-портал}~--- web-сайт, представляющий набор
      сервисов по одной или нескольким тематикам.
    \item[] \textbf{Токен авторизации}~--- зашифрованная
      последовательность символов, которая позволяет точно
      идентифицировать субъект в системе и определить уровень его
      привилегий.
  \end{itemize}

  \section*{Принятые сокращения}

  \begin{itemize}
    \item[] \textbf{ПО}~--- программное обеспечение.
    \item[] \textbf{СПО}~--- специальное программное обеспечение.
  \end{itemize}

  \chapter{Введение}

  Данное техническое задание составляется для проектирования СПО
  <<Система разграничения доступа к исходному коду лабораторных
  работ>>.
  Техническое задание выполняется в соответствии со стандартом ГОСТ
  19.201–78 <<ЕСПД. Техническое задание. Требования к содержанию и
  оформлению>>.

  \section{Краткое описание предметной области}

  В настоящее время в рамках лабораторных работ по курсу <<Основы
  искусственного интеллекта>> студентам кафедры ИУ7 предлагают
  разработать относительно небольшие программы, так как в противном
  случае у них не будет достаточно времени на их реализацию.
  Для того, чтобы в результате выполнения лабораторных работ студенты
  реализовывали более сложные программы за то же время, могут быть
  использованы технологии командной разработки.

  В настоящее время в качестве программных инструментов для выдачи и
  приёма лабораторных работ по курсу <<Основы искусственного
  интеллекта>> применяют web-портал GitLab, основанный на системе
  контроля версий Git.
  Однако при выдаче и приёме групповых лабораторных работ возникает
  проблема разграничения доступа к исходному коду на более низких
  уровнях, чем уровень репозиториев.

  Данное техническое задание определяет требования к разработке
  системы разграничения доступа студентов кафедры ИУ7 к исходному коду
  групповых лабораторных работ по курсу <<Основы искусственного
  интеллекта>>.

  \section{Существующие аналоги}

  В настоящее время для командной разработки ПО применяют web-порталы
  управления репозиториями, основанные на системе контроля версий Git:
  GitHub, GitLab, Gitea и т.д.
  Кроме того, существует web-портал управления репозиториями Darcs
  Hub, основанный на системе контроля версий Darcs.
  Данные web-порталы предоставляют разработчикам ПО следующие
  инструменты:
  \begin{itemize}[label=---]
    \item систему контроля версий разрабатываемого ПО;
    \item систему разграничения доступа разработчиков ПО к его
      исходному коду (на уровне репозиториев);
    \item систему документирования исходного кода разрабатываемого ПО;
    \item систему отслеживания ошибок в разрабатываемом ПО.
  \end{itemize}

  Основным преимуществом разрабатываемой системы перед аналогами
  должна являться возможность разграничения прав доступа к исходному
  коду на более низком уровне, чем уровень репозиториев.

  \section{Описание системы} \label{chapter:system-description}

  Разрабатываемая система должна предоставлять возможности выдачи и
  проверки групповых лабораторных работ с разграничением уровня
  доступа между студентами на усмотрение преподавателя.
  Каждый пользователь системы должен иметь одну из следующих ролей:
  \begin{itemize}[label=---]
    \item \textbf{администратор}~--- пользователь, осуществляющий
      регистрацию, изменение и удаление учётных записей преподавателей
      и студентов;
    \item \textbf{преподаватель}~--- пользователь, осуществляющий
      создание, изменение, удаление, выдачу и приём лабораторных
      работ;
    \item \textbf{студент}~--- пользователь, осуществляющий реализацию
      выданной ему части лабораторной работы.
  \end{itemize}

  Для обеспечения разграничения доступа студентов исходный код
  лабораторной работы может быть представлен в виде дерева
  (рисунок~\ref{fig:source-code-tree}).
  Помимо исходного кода, узлы такого дерева могут дополнительно
  содержать тесты и документацию.

  \begin{figure}[ht]
    \centering

    \begin{tikzpicture}
      \node{Лабораторная работа}
        child {
          node{Сервис 1}
            child {
              node{Компонент 1~~~~~}
                child {
                  node{Класс 1}
                    child {
                      node{Метод 1}
                    }
                    child {
                      node{\ldots}
                    }
                    child {
                      node{Метод N}
                    }
                }
                child {
                  node{\ldots}
                }
                child {
                  node{Класс N}
                }
            }
            child {
              node{\ldots}
            }
            child {
              node{~~~~~Компонент N}
            }
        }
        child {
          node{\ldots}
        }
        child {
          node{Сервис N}
        }
      ;
    \end{tikzpicture}

    \caption{Представление исходного кода в виде дерева}
    \label{fig:source-code-tree}
  \end{figure}

  Каждому узлу дерева преподаватель назначает владельца из числа
  студентов.
  Владелец узла должен иметь возможность:
  \begin{itemize}[label=---]
    \item просматривать и изменять соответствующий исходный код
      лабораторной работы;
    \item создавать дочерние узлы;
    \item предоставлять другим разработчикам возможность просматривать
      ассоциированный с узлом исходный код.
  \end{itemize}

  \section{Основания для разработки}

  Разработка ведётся в рамках выполнения лабораторных работ по курсу
  <<Методология программной инженерии>> на кафедре <<Программное
  обеспечение ЭВМ и информационные технологии>> факультета
  <<Информатика и системы управления>> МГТУ~им.~Н.~Э.~Баумана.

  \section{Назначение разработки}

  Назначение разрабатываемой системы~--- обеспечить возможность приёма
  и выдачи групповых лабораторных работ по курсу <<Основы
  искусственного интеллекта>> с разграничением доступа студентов к
  исходному коду на низких уровнях.

  Целевая аудитория разрабатываемой системы~--- студенты и
  преподаватели кафедры ИУ7 МГТУ~им.~Н.~Э.~Баумана, выполняющие и
  принимающие лабораторные работы по курсу <<Основы искусственного
  интеллекта>> соответственно.

  \section{Требования к системе}

  Разрабатываемая система должна обеспечивать:
  \begin{itemize}[label=---]
    \item хранение условий и исходного кода лабораторных работ на
      удалённых серверах;
    \item возможность создания локальных рабочих копий исходного кода
      лабораторных работ, хранящегося на удалённых серверах.
    \item возможность синхронизации локальных рабочих копий исходного
      кода лабораторных работ с хранящимся на удалённых серверах.
  \end{itemize}

  \subsection{Требования к функциональным характеристикам}

  Разрабатываемая система должна обеспечивать:
  \begin{itemize}[label=---]
    \item хранение не менее 1000 лабораторных работ, суммарный
      информационный объём исходного кода которых не превышает 1 ТБ;
    \item создание локальной рабочей копии лабораторной работы за
      время, не превышающее 15 секунд;
    \item синхронизацию локальной рабочей копии исходного кода
      лабораторных работ за время, не превышающее 15 секунд.
  \end{itemize}

  \subsection{Требования по реализации}

  Разрабатываемая система должна обеспечивать:
  \begin{itemize}[label=---]
    \item возможность независимого изменения студентами их узлов
      исходного кода лабораторной работы;
    \item web-интерфейс, позволяющий преподавателю осуществлять выдачу
      лабораторных работ;
    \item интерфейс командной строки, позволяющий студентам
      осуществлять создание локальных рабочих копий исходного кода
      лабораторных работ;
    \item взаимодействие клиентов с удалёнными серверами по протоколу
      HTTP в соответствии с принципами REST;
    \item деградацию функциональности или выдачу сообщения об ошибке в
      случае недоступности удалённых серверов;
    \item авторизацию пользователей как через web-интерфейс, так и
      через интерфейс командной строки;
    \item валидацию входных данных как на клиенте, так и на удалённом
      сервере.
  \end{itemize}

  \subsection{Функциональные требования с точки зрения пользователя}

  \subsubsection{Первоочередные}

  Система должна обеспечивать следующую функциональность при работе с пользователями.
  \begin{enumerate}[label*=\arabic*.]
    \item Регистрация пользователей с валидацией указанных
      сведений:
      \begin{itemize}[label=---]
        \item логин (уникальный в системе);
        \item пароль;
        \item адрес электронной почты (уникальный в системе).
      \end{itemize}
    \item Авторизация пользователей по логину и паролю.
    \item Выделение для пользователей следующих ролей в каждом из проектов: авторизованный пользователь, автор, владелец. Роль администратора не привязана к какому-либо проекту.
    \item Удаление пользователей (только для администратора).
  \end{enumerate}

  Система должна обеспечивать следующую функциональность при работе с проектами.
  \begin{enumerate}[label*=\arabic*.]
    \item Создание и регистрация проекта.
    \item Удаление проекта.
    \item Запрос списка участников определённого проекта.
    \item Запрос доступа к определённым частям проектов.
    \item Добавление изменений в проекты, в которых пользователь является автором.
    \item Запрос на добавление изменений в проекты, в которых пользователь не является автором.
    \item Возможность принятия или отклонения запроса на добавление изменений в проекты, в которых пользователь является автором.
    \item Управление правами доступа к исходному коду на языке Python в соответствии с уровнями, представленными на рисунке~\ref{fig:source-code-tree}, в остальных случаях регулировать доступ на уровне целых файлов.
    \item Управление ограничениями авторов более низкого уровня (ограничения не должы быть меньше управляющего ограничениями).
  \end{enumerate}

  Для разработчика платформа должен обеспечивать функциональность,
  описанную в разделе~\ref{chapter:system-description}.

  Для администратора платформа должна обеспечивать
  \begin{enumerate}[label*=\arabic*)]
    \item функциональность, доступную разработчику;
    \item возможность удалять проект.
  \end{enumerate}

  \subsubsection{Второочередные}

  Система должна обеспечивать следующую функциональность при работе с проектами.
  \begin{enumerate}[label*=\arabic*.]
    \item Работа в соответствии с уровнями, представленными на
      рисунке~\ref{fig:source-code-tree}, с кодами на всех языках
      программирования, задействованных в учебной программе кафедры
      ИУ7, такими как Bash, C, C++, JavaScript и т.д.
    \item Возможность запуска тестов в автоматическом режиме.
    \item Возможность обсуждения изменений, предлагаемых авторами.
    \item Возможность добавления комментариев и задач (issues),
      связанных с существующим исходным кодом проекта.
  \end{enumerate}

  \subsection{Входные параметры}

  Для неавторизованного пользователя:
  \begin{itemize}[label=---]
    \item идентификатор (уникальный в системе);
    \item пароль (строка из 10-32 символов base64).
  \end{itemize}

  Для авторизованного пользователя:
  \begin{itemize}[label=---]
    \item идентификатор (уникальный в системе);
    \item токен авторизации.
  \end{itemize}

  Для проекта:
  \begin{itemize}[label=---]
    \item идентификатор (уникальный в системе);
    \item название (строка из символов латинского алфавита, арабских
      цифр и знака <<->>, максимальная длина~--- 64 символа);
    \item структура каталогов лабораторной работы (максимальный
      объём~--- 1 ТБ).
  \end{itemize}

  \subsection{Выходные параметры}

  Для неавторизованного пользователя~--- токен авторизации.

  Для авторизованного пользователя:
  \begin{itemize}[label=---]
    \item список доступных пользователю лабораторных работ с их
      названиями, условиями и выполняющими студентами;
    \item дерево доступных пользователю узлов исходного кода каждой
      лабораторной работы.
  \end{itemize}

  \subsection{Топология системы}

  \includeimage
    {topology}
    {f}
    {ht}
    {0.9 \textwidth}
    {Топология системы}

  \subsection{Требования к составу и параметрам технических средств}

  Все клиентские и серверные приложения должны потреблять суммарно не
  более 4 Гб оперативной памяти и работать на машине с архитектурой
  процессора amd64 или arm64, а также под управлением ОС Windows (10,
  11), Linux (версия ядра 5.15 и выше).

  \subsection{Сценарии функционирования системы}

  \includeimage
    {usecase}
    {f}
    {ht}
    {0.9 \textwidth}
    {Диаграмма прецедентов}

  \subsection{Требования к надёжности}

  Система должна работать в соответствии с данным техническим заданием
  независимо от поведения пользователей и связности сети, в которой
  они работают.
  Студенты должны иметь возможность изолированно работать над своими
  частями лабораторных работ, расположенными на локальных машинах.

  \section{Требования к документации}

  Исполнитель должен подготовить и передать заказчику следующие
  документы:
  \begin{itemize}[label=---]
    \item руководство по развёртыванию системы;
    \item руководство по администрированию системы;
    \item руководство по эксплуатации системы студентами и
      преподавателями.
  \end{itemize}
\end{document}
