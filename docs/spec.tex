\documentclass{bmstu}

\usepackage{placeins}

\begin{document}
  \chapter*{Система разграничения доступа к исходному коду
  лабораторных работ}

  \section*{Глоссарий}

  \begin{itemize}
    \item[] \textbf{Валидация данных}~--- проверка данных на
      соответствие заданным условиям и ограничениям.
    \item[] \textbf{Web-портал}~--- web-сайт, представляющий набор
      сервисов по одной или нескольким тематикам.
    \item[] \textbf{Токен авторизации}~--- зашифрованная
      последовательность символов, которая позволяет точно
      идентифицировать субъект в системе и определить уровень его
      привилегий.
  \end{itemize}

  \section*{Принятые сокращения}

  \begin{itemize}
    \item[] \textbf{ПО}~--- программное обеспечение.
    \item[] \textbf{СПО}~--- специальное программное обеспечение.
  \end{itemize}

  \chapter{Введение}

  Данное техническое задание составляется для проектирования СПО
  <<Система разграничения доступа к исходному коду лабораторных
  работ>>.
  Техническое задание выполняется в соответствии со стандартом ГОСТ
  19.201–78 <<ЕСПД. Техническое задание. Требования к содержанию и
  оформлению>>.

  \section{Краткое описание предметной области}

  В настоящее время в рамках лабораторных работ по курсу <<Основы
  искусственного интеллекта>> студентам кафедры ИУ7 предлагают
  разработать относительно небольшие программы, так как в противном
  случае у них не будет достаточно времени на их реализацию.
  Для того, чтобы в результате выполнения лабораторных работ студенты
  реализовывали более сложные программы за то же время, могут быть
  использованы технологии командной разработки.

  В настоящее время в качестве программных инструментов для выдачи и
  приёма лабораторных работ по курсу <<Основы искусственного
  интеллекта>> применяют web-портал GitLab, основанный на системе
  контроля версий Git.
  Однако при выдаче и приёме групповых лабораторных работ возникает
  проблема разграничения доступа к исходному коду на более низких
  уровнях, чем уровень репозиториев.

  Данное техническое задание определяет требования к разработке
  системы разграничения доступа студентов кафедры ИУ7 к исходному коду
  групповых лабораторных работ по курсу <<Основы искусственного
  интеллекта>>.

  \section{Существующие аналоги}

  В настоящее время для командной разработки ПО применяют web-порталы
  управления репозиториями, основанные на системе контроля версий Git:
  GitHub, GitLab, Gitea и т.д.
  Кроме того, существует web-портал управления репозиториями Darcs
  Hub, основанный на системе контроля версий Darcs.
  Данные web-порталы предоставляют разработчикам ПО следующие
  инструменты:
  \begin{itemize}[label=---]
    \item систему контроля версий разрабатываемого ПО;
    \item систему разграничения доступа разработчиков ПО к его
      исходному коду (на уровне репозиториев);
    \item систему документирования исходного кода разрабатываемого ПО;
    \item систему отслеживания ошибок в разрабатываемом ПО.
  \end{itemize}

  Основным преимуществом разрабатываемой системы перед аналогами
  должна являться возможность разграничения прав доступа к исходному
  коду на более низком уровне, чем уровень репозиториев.

  \section{Описание системы} \label{chapter:system-description}

  Разрабатываемая система должна предоставлять возможности выдачи и
  проверки групповых лабораторных работ с разграничением уровня
  доступа между студентами на усмотрение преподавателя.
  Каждый пользователь системы должен иметь одну из следующих ролей:
  \begin{itemize}[label=---]
    \item \textbf{администратор}~--- пользователь, осуществляющий
      регистрацию, изменение и удаление учётных записей преподавателей
      и студентов;
    \item \textbf{преподаватель}~--- пользователь, осуществляющий
      создание, изменение, удаление, выдачу и приём лабораторных
      работ;
    \item \textbf{студент}~--- пользователь, осуществляющий реализацию
      выданной ему части лабораторной работы.
  \end{itemize}

  Для обеспечения разграничения доступа студентов исходный код
  лабораторной работы может быть представлен в виде дерева
  (рисунок~\ref{fig:source-code-tree}).
  Помимо исходного кода, узлы такого дерева могут дополнительно
  содержать тесты и документацию.

  \begin{figure}[ht]
    \centering

    \begin{tikzpicture}
      \node{Лабораторная работа}
        child {
          node{Сервис 1}
            child {
              node{Компонент 1~~~~~}
                child {
                  node{Класс 1}
                    child {
                      node{Метод 1}
                    }
                    child {
                      node{\ldots}
                    }
                    child {
                      node{Метод N}
                    }
                }
                child {
                  node{\ldots}
                }
                child {
                  node{Класс N}
                }
            }
            child {
              node{\ldots}
            }
            child {
              node{~~~~~Компонент N}
            }
        }
        child {
          node{\ldots}
        }
        child {
          node{Сервис N}
        }
      ;
    \end{tikzpicture}

    \caption{Представление исходного кода в виде дерева}
    \label{fig:source-code-tree}
  \end{figure}

  Каждому узлу дерева преподаватель назначает владельца из числа
  студентов.
  Владелец узла должен иметь возможность:
  \begin{itemize}[label=---]
    \item просматривать и изменять соответствующий исходный код
      лабораторной работы;
    \item создавать дочерние узлы;
    \item предоставлять другим разработчикам возможность просматривать
      ассоциированный с узлом исходный код.
  \end{itemize}

  \section{Основания для разработки}

  Разработка ведётся в рамках выполнения лабораторных работ по курсу
  <<Методология программной инженерии>> на кафедре <<Программное
  обеспечение ЭВМ и информационные технологии>> факультета
  <<Информатика и системы управления>> МГТУ~им.~Н.~Э.~Баумана.

  \section{Назначение разработки}

  Назначение разрабатываемой системы~--- обеспечить возможность приёма
  и выдачи групповых лабораторных работ по курсу <<Основы
  искусственного интеллекта>> с разграничением доступа студентов к
  исходному коду на низких уровнях.

  Целевая аудитория разрабатываемой системы~--- студенты и
  преподаватели кафедры ИУ7 МГТУ~им.~Н.~Э.~Баумана, выполняющие и
  принимающие лабораторные работы по курсу <<Основы искусственного
  интеллекта>> соответственно.

  \section{Требования к системе}

  \begin{enumerate}[label*=\arabic*.]
    \item Разрабатываемая система должна обеспечивать возможность
      работы как с проектами, расположенными на удалённых серверах,
      так и с их копиями, располагающимися на устройствах
      пользователей.
    \item Система должна обеспечивать децентрализацию хранения
      различных версий проектов. При этом все версии проектов должны
      синхронизироваться с сервером владельца (правообладателя)
      проекта.
    \item До синхронизации версия проекта считается локальной и не
      доступна другим пользователям.
    \item Система должна обеспечивать хранение следующих составляющих
      программных проектов:
    \begin{itemize}[label*=---]
      \item исходный код всех разработчиков проекта;
      \item наборы данных для обучения моделей;
      \item эксплуатационная документация (относится ко всему проекту
        в целом);
      \item внутренняя документация (относится к отдельным
        составляющим исходного кода всего проекта).
    \end{itemize}
    \item Система должна обеспечить соблюдение разграничения прав
      доступа к частям проектов, заложенное его авторами.
  \end{enumerate}

  \subsection{Требования к функциональным характеристикам}

  \begin{enumerate}[label*=\arabic*.]
    \item Время отклика системы на запрос фиксации новой версии
      проекта не должно превышать 5 секунд.
    \item Обеспечить заданные временные характеристики для проектов,
      размер исходного кода которых не превышает 1 ТБ, а количество
      файлов не превышает 1000.
  \end{enumerate}

  \subsection{Требования по реализации}

  \begin{enumerate}[label*=\arabic*.]
  \item Все пользователи системы должны работать с различными версиями узлов проектов изолированно друг от друга.
  \item Необходимо реализовать интерфейс участника проекта, позволяющий синхронизировать версии проекта с его владельцем (отправлять изменения
  узлов проекта владельцу).
  \item Необходимо реализовать интерфейс пользователя с использованием командной строки (консоли), позволяющий получать информацию о доступных проектах и работать с версиями проектов на локальной машине.
  \item Система поставляется в виде одного исполняемого файла, содержащего интерфейс командной строки для работы с проектами на устройстве пользователя, а также web-интерфейс для взаимодействия с другими пользователями системы.
  \item При необходимости, каждый проект имеет своё собственное хранилище версий на машине каждого его пользователя.
  \item Для синхронизации данных о проектах между пользователями использовать протокол HTTP (придерживаться RESTful).
  \item При недоступности серверов владельцев проектов должна осуществляться деградация функционала или выдача пользователю сообщения об ошибке.
  \item Необходимо предусмотреть авторизацию пользователей как через web-интерфейс приложения, так и через консольный интерфейс.
  \item Валидацию входных данных необходимо проводить и на стороне участника проекта, и на стороне его владельца.
  \end{enumerate}

  \subsection{Функциональные требования с точки зрения пользователя}

  \subsubsection{Первоочередные}

  Система должна обеспечивать следующую функциональность при работе с пользователями.
  \begin{enumerate}[label*=\arabic*.]
    \item Регистрация пользователей с валидацией указанных
      сведений:
      \begin{itemize}[label=---]
        \item логин (уникальный в системе);
        \item пароль;
        \item адрес электронной почты (уникальный в системе).
      \end{itemize}
    \item Авторизация пользователей по логину и паролю.
    \item Выделение для пользователей следующих ролей в каждом из проектов: авторизованный пользователь, автор, владелец. Роль администратора не привязана к какому-либо проекту.
    \item Удаление пользователей (только для администратора).
  \end{enumerate}

  Система должна обеспечивать следующую функциональность при работе с проектами.
  \begin{enumerate}[label*=\arabic*.]
    \item Создание и регистрация проекта.
    \item Удаление проекта.
    \item Запрос списка участников определённого проекта.
    \item Запрос доступа к определённым частям проектов.
    \item Добавление изменений в проекты, в которых пользователь является автором.
    \item Запрос на добавление изменений в проекты, в которых пользователь не является автором.
    \item Возможность принятия или отклонения запроса на добавление изменений в проекты, в которых пользователь является автором.
    \item Управление правами доступа к исходному коду на языке Python в соответствии с уровнями, представленными на рисунке~\ref{fig:source-code-tree}, в остальных случаях регулировать доступ на уровне целых файлов.
    \item Управление ограничениями авторов более низкого уровня (ограничения не должы быть меньше управляющего ограничениями).
  \end{enumerate}

  Для разработчика платформа должен обеспечивать функциональность,
  описанную в разделе~\ref{chapter:system-description}.

  Для администратора платформа должна обеспечивать
  \begin{enumerate}[label*=\arabic*)]
    \item функциональность, доступную разработчику;
    \item возможность удалять проект.
  \end{enumerate}

  \subsubsection{Второочередные}

  Система должна обеспечивать следующую функциональность при работе с проектами.
  \begin{enumerate}[label*=\arabic*.]
    \item Работа в соответствии с уровнями, представленными на
      рисунке~\ref{fig:source-code-tree}, с кодами на всех языках
      программирования, задействованных в учебной программе кафедры
      ИУ7, такими как Bash, C, C++, JavaScript и т.д.
    \item Возможность запуска тестов в автоматическом режиме.
    \item Возможность обсуждения изменений, предлагаемых авторами.
    \item Возможность добавления комментариев и задач (issues),
      связанных с существующим исходным кодом проекта.
  \end{enumerate}

  \subsection{Входные параметры}

  Для незарегистрированного пользователя:
  \begin{enumerate}[label*=\arabic*)]
    \item логин (строка из символов base64, максимальная длина~--- 64
      символа);
    \item пароль (строка из 10-32 символов base64);
    \item адрес электронной почты (до 64 символов).
  \end{enumerate}

  Для неавторизованного пользователя:
  \begin{enumerate}[label*=\arabic*)]
    \item идентификатор (уникальный в системе);
    \item пароль (строка из 10-32 символов base64);
  \end{enumerate}

  Для авторизованного пользователя:
  \begin{enumerate}[label*=\arabic*)]
    \item идентификатор (уникальный в системе);
    \item токен авторизации;
  \end{enumerate}

  Для проекта:
  \begin{enumerate}[label*=\arabic*)]
    \item идентификатор (уникальный в системе);
    \item название (строка из символов латинского алфавита, арабских
      цифр и знака <<->>, максимальная длина~--- 64 символа).
    \item структура каталогов репозитория (максимальный объём~--- 1
      ТБ).
  \end{enumerate}

  \subsection{Выходные параметры}

  Для неавторизованного пользователя~--- токен авторизации.

  Для авторизованного пользователя:
  \begin{enumerate}[label*=\arabic*)]
    \item список доступных пользователю проектов с их названиями,
      именем владельца и списком авторов;
    \item дерево доступных пользователю узлов исходного кода для
      каждого проекта.
  \end{enumerate}

  \subsection{Топология системы}

  \includeimage
    {topology}
    {f}
    {ht}
    {0.9 \textwidth}
    {Топология системы}

  \subsection{Требования к составу и параметрам технических средств}

  Все клиентские и серверные приложения должны потреблять суммарно не
  более 4Гб оперативной памяти и работать на машине с архитектурой
  процессора amd64 или arm64, а также под управлением ОС Windows (10,
  11), Linux (версия ядра 5.15 и выше).

  \subsection{Сценарии функционирования системы}

  \includeimage
    {usecase}
    {f}
    {ht}
    {0.9 \textwidth}
    {Диаграмма прецедентов}

  \subsection{Требования к надёжности}

  Система должна работать в соответствии с данным техническим заданием
  независимо от поведения владельцев проектов и связности сети, в
  которой работают их участники.
  Пользователи должны иметь возможность изолированно работать над
  своими проектами, расположенными на локальной машине.

  \section{Требования к документации}

  Исполнитель должен подготовить и передать заказчику следующие
  документы:
  \begin{enumerate}[label*=\arabic*)]
    \item руководство по развёртыванию системы;
    \item руководство пользователя системы;
    \item руководство автора проекта;
    \item руководство администратора системы.
  \end{enumerate}
\end{document}
