\documentclass{bmstu}

\usepackage{placeins}

\begin{document}
  \chapter{Платформа для поддержки разработки и внедрения программного
  обеспечения}

  \section*{Глоссарий}

  \begin{itemize}
    \item[] \textbf{Валидация данных}~--- проверка данных на
      соответствие заданным условиям и ограничениям.
    \item[] \textbf{Портал}~--- Web-сайт, представляющий набор
      сервисов по одной или нескольким тематикам.
    \item[] \textbf{Платформа}~--- сайт или программное обеспечение,
      которое позволяет пользователям совершать с контентом различные
      действия (получать, производить, размещать, распространять,
      хранить).
    \item[] \textbf{Репозиторий}~--- хранилище для исходного кода
      программного продукта и для его сопровождающих материалов, к
      которым относятся тесты, документация, наборы данных для
      обучения моделей и т.д.
    \item[] \textbf{Автор}~--- человек, оказавший в том или ином
      качестве услугу в достижении результатов исследования,
      достаточно существенную, чтобы быть признанным разработчиком.
    \item[] \textbf{Ограничения автора}~--- запреты на определенный
      функционал в системе.
    \item[] \textbf{Уровень автора}~--- степень максимального узла
      дерева, к которому имеет доступ автор. Автор максимального
      уровня имеет доступ к корневому узлу проекта.
    \item[] \textbf{Токен авторизации}~--- зашифрованная
      последовательность символов, которая позволяет точно
      идентифицировать субъект в системе и определить уровень его
      привилегий.
  \end{itemize}

  \section*{Принятые сокращения}

  \begin{itemize}
    \item[] \textbf{ПО}~--- программное обеспечение.
    \item[] \textbf{СПО}~--- специальное программное обеспечение.
  \end{itemize}

  \section*{Введение}

  Данное техническое задание составляется для проектирования СПО
  «Платформа для поддержки разработки и распространения программного
  обеспечения».
  Техническое задание выполняется в соответствии со стандартом ГОСТ
  19.201–78 <<ЕСПД. Техническое задание. Требования к содержанию и
  оформлению>>.

  \section{Краткое описание предметной области}

  Существует проблема регистрации программных проектов как
  самостоятельных представителей определённой технологии и проблема
  разграничения прав доступа разработчиков к исходному коду на более
  низких уровнях, чем уровень репозиториев.
  Решение данных проблем позволило бы осуществлять поиск и интеграцию
  зарегистрированных проектов в другие проекты и устанавливать факт
  того, что разработчик является автором определенной части проекта
  или автором проекта, использующего зарегистрированное на платформе
  техническое решение.

  Сервисами, предоставляющими инструментарий для разработки ПО,
  являются системы контроля версий, системы поддержки разработки
  проектов или CASE-технологии.
  Такие сервисы зачастую являются порталами, с списком доступных для
  пользователя технологических решений и возможностью поиска как по
  названию, так и по различным характеристикам доступных для чтения
  проектов.

  Данное техническое задание определяет требования к разработке
  портала для поддержки разработки и распространения ПО.

  \section{Существующие аналоги}

  В настоящее время для совместной разработки ПО применяют порталы
  управления репозиториями, основанные на системе контроля версий Git:
  GitHub, GitLab, Gitea и т.д.
  Кроме того, существует портал управления репозиториями Darcs Hub,
  основанный на системе контроля версий Darcs.
  Данные порталы предоставляют разработчикам ПО следующие инструменты:
  \begin{itemize}[label=---]
    \item систему контроля версий разрабатываемого ПО;
    \item систему разграничения прав доступа разработчиков ПО к его
      исходному коду (на уровне репозиториев);
    \item систему документирования исходного кода разрабатываемого ПО;
    \item систему отслеживания ошибок в разрабатываемом ПО.
  \end{itemize}

  Разрабатываемая система должна иметь следующие преимущества перед
  аналогами:
  \begin{itemize}[label=---]
    \item Возможность установления связи между зарегистрированными
      проектами, что позволит определить стек технологий,
      задействованных для разработки программных комплексов.
    \item Возможность для автора проекта управлять доступом к его
      деталям, давая определённым пользователям доступ к исходному
      коду отдельных компонентов ПО.
  \end{itemize}

  \section{Описание системы} \label{chapter:system-description}

  Разрабатываемая система должна представлять собой платформу для
  создания, регистрации, поддержки разработки, внедрения и
  модернизации программных проектов.
  Система может выдавать пользователям следующие роли.
  \begin{enumerate}[label*=\arabic*.]
    \item \textbf{Авторизованный пользователь}~--- пользователь,
      который имеет ограниченный доступ к проекту.
    \item \textbf{Автор проекта}~--- пользователь, который имеет
      ограниченный доступ к проекту и вносит свои изменения в
      доступные ему части проекта. Ограничения автора могут быть
      настроены владельцем проекта как в момент приглашения
      пользователя в проект, так и впоследствии. Ограничения могут
      быть наложены на такие функции, как приглашение сторонних
      разработчиков в проект, внесение изменений в определённые части
      проекта, возможность принимать или отклонять изменения от других
      пользователей.
    \item \textbf{Владелец проекта}~--- пользователь, который создал
      проект и предоставляет доступ к нему другим пользователям через
      web-сервер, работающий на его машине. Также владелец может
      добавлять и удалять ограничения у любых авторов своего проекта.
    \item \textbf{Администратор системы}~--- пользователь, который
      имеет права на удаление других пользователей системы, а также на
      изменение их ролей в проектах.
  \end{enumerate}

  Владелец может создать новый проект, сделав его либо полностью
  открытым для всех пользователей платформы, либо закрытым, оставив
  лишь демонстрацию, в которой будут кратко описаны цели проекта, его
  основные достижения и использованные технологии.
  Владелец является правообладателем всех версий проекта и может
  изменять роли пользователей в нём.

  Пользователи, которым доступен просмотр содержимого проекта, могут
  предлагать свои изменения.
  При внесении изменений в проект пользователь становится его
  \textbf{субавтором}.
  Если его изменения будут приняты авторами или владельцем проекта, то
  пользователь должен стать автором и получить доступ к частям
  проекта, которые он модифицировал.
  Использование зарегистрированных проектов в других проектах будет
  сопровождаться обратной связью для авторов использованных
  технологических решений.

  Для обеспечения разграничения прав доступа разработчиков ПО его
  исходный код может быть представлен в виде дерева
  (рисунок~\ref{fig:source-code-tree}).

  \begin{figure}[ht]
    \centering

    \begin{tikzpicture}
      \node{Проект}
        child {
          node{Сервис 1}
            child {
              node{Компонент 1~~~~~}
                child {
                  node{Класс 1}
                    child {
                      node{Метод 1}
                    }
                    child {
                      node{\ldots}
                    }
                    child {
                      node{Метод N}
                    }
                }
                child {
                  node{\ldots}
                }
                child {
                  node{Класс N}
                }
            }
            child {
              node{\ldots}
            }
            child {
              node{~~~~~Компонент N}
            }
        }
        child {
          node{\ldots}
        }
        child {
          node{Сервис N}
        }
      ;
    \end{tikzpicture}

    \caption{Представление исходного кода в виде дерева}
    \label{fig:source-code-tree}
  \end{figure}

  Помимо исходного кода ПО пользователь может загружать тесты,
  документацию и наборы данных для обучения моделей.
  Для них права доступа должны регулироваться на уровне файлов.

  В таком случае у каждого узла этого дерева должен быть разработчик,
  который им владеет.
  Владелец узла должен иметь возможность:
  \begin{itemize}[label=---]
    \item просматривать и изменять соответствующий исходный код
      разрабатываемого ПО;
    \item создавать дочерние узлы;
    \item предоставлять другим разработчикам возможность просматривать
      и/или изменять ассоциированный с узлом исходный код;
    \item передавать владение узлом другим разработчикам.
  \end{itemize}

  \section{Основания для разработки}

  Разработка ведётся в рамках выполнения лабораторных работ по курсу
  <<Методология программной инженерии>> на кафедре <<Программное
  обеспечение ЭВМ и информационные технологии>> факультета
  <<Информатика и системы управления>> МГТУ~им.~Н.~Э.~Баумана.

  \section{Назначение разработки}

  Назначение разрабатываемой платформы~--- обеспечить возможность
  совместной разработки программных проектов в сфере машинного
  обучения с разграничением прав доступа разработчиков к исходному
  коду на низких уровнях.

  Целевая аудитория~--- студенты и преподаватели кафедры ИУ7
  \mbox{МГТУ~им.~Н.~Э.~Баумана}.

  \section{Требования к системе}
  \begin{enumerate}[label*=\arabic*.]
    \item Разрабатываемая система должна обеспечивать возможность
      работы как с проектами, расположенными на удалённых серверах,
      так и с их копиями, располагающимися на устройствах
      пользователей.
    \item Система должна обеспечивать децентрализацию хранения
      различных версий проектов. При этом все версии проектов должны
      синхронизироваться с сервером владельца (правообладателя)
      проекта.
    \item До синхронизации версия проекта считается локальной и не
      доступна другим пользователям.
    \item Система должна обеспечивать хранение следующих составляющих
      программных проектов:
    \begin{itemize}[label*=---]
      \item исходный код всех разработчиков проекта;
      \item наборы данных для обучения моделей;
      \item эксплуатационная документация (относится ко всему проекту
        в целом);
      \item внутренняя документация (относится к отдельным
        составляющим исходного кода всего проекта).
    \end{itemize}
    \item Система должна обеспечить соблюдение разграничения прав
      доступа к частям проектов, заложенное его авторами.
  \end{enumerate}

  \subsection{Требования к функциональным характеристикам}
  \begin{enumerate}[label*=\arabic*.]
    \item Время отклика системы на запрос фиксации новой версии
      проекта не должно превышать 5 секунд.
    \item Обеспечить заданные временные характеристики для проектов,
      размер исходного кода которых не превышает 1 ТБ, а количество
      файлов не превышает 1000.
  \end{enumerate}

  \subsection{Требования по реализации}

  \begin{enumerate}[label*=\arabic*.]
  \item Все пользователи системы должны работать с различными версиями узлов проектов изолированно друг от друга.
  \item Необходимо реализовать интерфейс участника проекта, позволяющий синхронизировать версии проекта с его владельцем (отправлять изменения
  узлов проекта владельцу).
  \item Необходимо реализовать интерфейс пользователя с использованием командной строки (консоли), позволяющий получать информацию о доступных проектах и работать с версиями проектов на локальной машине.
  \item Система поставляется в виде одного исполняемого файла, содержащего интерфейс командной строки для работы с проектами на устройстве пользователя, а также web-интерфейс для взаимодействия с другими пользователями системы.
  \item При необходимости, каждый проект имеет своё собственное хранилище версий на машине каждого его пользователя.
  \item Для синхронизации данных о проектах между пользователями использовать протокол HTTP (придерживаться RESTful).
  \item При недоступности серверов владельцев проектов должна осуществляться деградация функционала или выдача пользователю сообщения об ошибке.
  \item Необходимо предусмотреть авторизацию пользователей как через web-интерфейс приложения, так и через консольный интерфейс.
  \item Валидацию входных данных необходимо проводить и на стороне участника проекта, и на стороне его владельца.
  \end{enumerate}

  \subsection{Функциональные требования с точки зрения пользователя}

  \subsubsection{Первоочередные}

  Система должна обеспечивать следующую функциональность при работе с пользователями.
  \begin{enumerate}[label*=\arabic*.]
    \item Регистрация пользователей с валидацией указанных
      сведений:
      \begin{itemize}[label=---]
        \item логин (уникальный в системе);
        \item пароль;
        \item адрес электронной почты (уникальный в системе).
      \end{itemize}
    \item Авторизация пользователей по логину и паролю.
    \item Выделение для пользователей следующих ролей в каждом из проектов: авторизованный пользователь, автор, владелец. Роль администратора не привязана к какому-либо проекту.
    \item Удаление пользователей (только для администратора).
  \end{enumerate}

  Система должна обеспечивать следующую функциональность при работе с проектами.
  \begin{enumerate}[label*=\arabic*.]
    \item Создание и регистрация проекта.
    \item Удаление проекта.
    \item Запрос списка участников определённого проекта.
    \item Запрос доступа к определённым частям проектов.
    \item Добавление изменений в проекты, в которых пользователь является автором.
    \item Запрос на добавление изменений в проекты, в которых пользователь не является автором.
    \item Возможность принятия или отклонения запроса на добавление изменений в проекты, в которых пользователь является автором.
    \item Управление правами доступа к исходному коду на языке Python в соответствии с уровнями, представленными на рисунке~\ref{fig:source-code-tree}, в остальных случаях регулировать доступ на уровне целых файлов.
    \item Управление ограничениями авторов более низкого уровня (ограничения не должы быть меньше управляющего ограничениями).
  \end{enumerate}

  Для разработчика платформа должен обеспечивать функциональность,
  описанную в разделе~\ref{chapter:system-description}.

  Для администратора платформа должна обеспечивать
  \begin{enumerate}[label*=\arabic*)]
    \item функциональность, доступную разработчику;
    \item возможность удалять проект.
  \end{enumerate}

  \subsubsection{Второочередные}

  Система должна обеспечивать следующую функциональность при работе с проектами.
  \begin{enumerate}[label*=\arabic*.]
    \item Работа в соответствии с уровнями, представленными на
      рисунке~\ref{fig:source-code-tree}, с кодами на всех языках
      программирования, задействованных в учебной программе кафедры
      ИУ7, такими как Bash, C, C++, JavaScript и т.д.
    \item Возможность запуска тестов в автоматическом режиме.
    \item Возможность обсуждения изменений, предлагаемых авторами.
    \item Возможность добавления комментариев и задач (issues),
      связанных с существующим исходным кодом проекта.
  \end{enumerate}

  \subsection{Входные параметры}

  Для незарегистрированного пользователя:
  \begin{enumerate}[label*=\arabic*)]
    \item логин (строка из символов base64, максимальная длина~--- 64
      символа);
    \item пароль (строка из 10-32 символов base64);
    \item адрес электронной почты (до 64 символов).
  \end{enumerate}

  Для неавторизованного пользователя:
  \begin{enumerate}[label*=\arabic*)]
    \item идентификатор (уникальный в системе);
    \item пароль (строка из 10-32 символов base64);
  \end{enumerate}

  Для авторизованного пользователя:
  \begin{enumerate}[label*=\arabic*)]
    \item идентификатор (уникальный в системе);
    \item токен авторизации;
  \end{enumerate}

  Для проекта:
  \begin{enumerate}[label*=\arabic*)]
    \item идентификатор (уникальный в системе);
    \item название (строка из символов латинского алфавита, арабских
      цифр и знака <<->>, максимальная длина~--- 64 символа).
    \item структура каталогов репозитория (максимальный объём~--- 1
      ТБ).
  \end{enumerate}

  \subsection{Выходные параметры}

  Для неавторизованного пользователя~--- токен авторизации.

  Для авторизованного пользователя:
  \begin{enumerate}[label*=\arabic*)]
    \item список доступных пользователю проектов с их названиями,
      именем владельца и списком авторов;
    \item дерево доступных пользователю узлов исходного кода для
      каждого проекта.
  \end{enumerate}

  \subsection{Топология системы}

  \includeimage
    {topology}
    {f}
    {ht}
    {0.9 \textwidth}
    {Топология системы}

  \subsection{Требования к составу и параметрам технических средств}

  Все клиентские и серверные приложения должны потреблять суммарно не
  более 4Гб оперативной памяти и работать на машине с архитектурой
  процессора amd64 или arm64, а также под управлением ОС Windows (10,
  11), Linux (версия ядра 5.15 и выше).

  \subsection{Сценарии функционирования системы}

  \includeimage
    {usecase}
    {f}
    {ht}
    {0.9 \textwidth}
    {Диаграмма прецедентов}

  \subsection{Требования к надёжности}

  Система должна работать в соответствии с данным техническим заданием
  независимо от поведения владельцев проектов и связности сети, в
  которой работают их участники.
  Пользователи должны иметь возможность изолированно работать над
  своими проектами, расположенными на локальной машине.

  \section{Требования к документации}

  Исполнитель должен подготовить и передать заказчику следующие
  документы:
  \begin{enumerate}[label*=\arabic*)]
    \item руководство по развёртыванию системы;
    \item руководство пользователя системы;
    \item руководство автора проекта;
    \item руководство администратора системы.
  \end{enumerate}
\end{document}
